%----------------------------------------------------------------------------------------
%	PACKAGES AND OTHER DOCUMENT CONFIGURATIONS
%----------------------------------------------------------------------------------------

\documentclass[12pt]{article}

\usepackage{polski}
\usepackage{ucs}
\usepackage[polish]{babel} 
\usepackage[utf8x]{inputenc}
\usepackage{graphicx}
\usepackage{color} 
\usepackage{listings} 
\usepackage{geometry} 
\geometry{
 	a4paper, 
 	left	=20mm,
 	right	=20mm,
 	top		=20mm,
 	bottom	=20mm, 
} 

\usepackage{tabularx}
\usepackage{hhline}
\usepackage{multirow}
 
\definecolor{codegreen}{rgb}{0,0.6,0}
\definecolor{codegray}{rgb}{0.5,0.5,0.5}
\definecolor{codepurple}{rgb}{0.58,0,0.82}
\definecolor{backcolour}{rgb}{0.95,0.95,0.92}
 
\lstdefinestyle{mystyle}{
    backgroundcolor=\color{backcolour},   
    commentstyle=\color{codegreen},
    keywordstyle=\color{magenta},
    numberstyle=\tiny\color{codegray},
    stringstyle=\color{codepurple},
    basicstyle=\tiny,
    breakatwhitespace=false,         
    breaklines=true,                 
    captionpos=b,                    
    keepspaces=true,                 
    numbers=left,                    
    numbersep=5pt,                  
    showspaces=false,                
    showstringspaces=false,
    showtabs=false,                  
    tabsize=2  
}
 
%----------------------------------------------------------------------------------------

\begin{document}

\lstset{
    language=VHDL,
    inputencoding=utf8x, 
    extendedchars=\true,
    basicstyle=\scriptsize,
    literate=
            {ą}{{\k{a}}}1
            {Ą}{{\k{A}}}1
            {ę}{{\k{e}}}1
            {Ę}{{\k{E}}}1
            {ó}{{\'o}}1
            {Ó}{{\'O}}1
            {ś}{{\'s}}1
            {Ś}{{\'S}}1
            {ł}{{\l{}}}1
            {Ł}{{\L{}}}1
            {ż}{{\.z}}1
            {Ż}{{\.Z}}1
            {ź}{{\'z}}1
            {Ź}{{\'Z}}1
            {ć}{{\'c}}1
            {Ć}{{\'C}}1
            {ń}{{\'n}}1
            {Ń}{{\'N}}1
}

\begin{table}[h]
\centering
\begin{tabularx}{\textwidth}{|c|c|c|X|}
\hline
\multicolumn{4}{|c|}{
\begin{tabular}{ccc}
\begin{tabular}{c}
\includegraphics[height=2.2cm]{../res/img/agh_logo_color.png}\\
\end{tabular}
&
\begin{tabular}{c}
\\[-10pt]
\large{Akademia Górniczo-Hutnicza w~Krakowie}\\[5pt]
\textsc{Wydział Elektrotechniki, Automatyki,}\\
\textsc{Informatyki i~Inżynierii Biomedycznej}\\[5pt]
\textsc{Katedra Automatyki i Inżynierii Biomedycznej}
\end{tabular}
&
\begin{tabular}{c}
\includegraphics[height=2.2cm]{../res/img/eaiie_logo.png}\\
\end{tabular}
\end{tabular}
}\\
\hline
\multicolumn{4}{|l|}{\scriptsize{Laboratorium:}}\\[-5pt]
\multicolumn{4}{|c|}{\textbf{Technika cyfrowa i~mikroprocesorowa –~semestr
zimowy}}\\
\multicolumn{4}{|c|}{(Elektronika cyfrowa)}\\[3pt]
\hline
\multicolumn{3}{|l|}{\scriptsize{Temat:}}
&
\multicolumn{1}{l|}{\scriptsize{Grupa:}}\\
\multicolumn{3}{|c|}{\textbf{VHDL - obsługa enkodera inkrementalnego}}
&
\multicolumn{1}{c|}{\textbf{wtorek 12\textsuperscript{30}}}\\
\hline
\multicolumn{2}{|l|}{
\begin{tabularx}{225pt}{X}
\scriptsize{Nazwisko i~imię:}\\
\multicolumn{1}{c}{\textbf{Maciejewski Michał}}
\end{tabularx}}
&
\multicolumn{2}{l|}{
\begin{tabularx}{225pt}{X}
\scriptsize{Nazwisko i~imię:}\\
\multicolumn{1}{c}{\textbf{Adasiewicz Konrad}}
\end{tabularx}}\\
\hline
\multicolumn{2}{|l|}{
\begin{tabularx}{225pt}{X}
\scriptsize{Pierwsze zajęcia:}\\
\multicolumn{1}{c}{\textbf{25.11.2014}}
\end{tabularx}}
&
\multicolumn{2}{l|}{
\begin{tabularx}{225pt}{X}
\scriptsize{Drugie zajęcia:}\\
\multicolumn{1}{c}{\textbf{02.12.2014}}
\end{tabularx}}\\
\hline
\end{tabularx}
\end{table}

\begin{lstlisting}[language=VHDL, style=mystyle]
-- Licznik 3xBCD będący pośrednikiem pomiędzy wyświetlaczem a
-- strukturami go taktującymi(licznik mod 4 oraz algorytmy enkoderowe).
-- Licznik zmienia swoją wartość na podstawie sygnału CNT_VAL, który
-- jest 2-bitowym wektorem, i przepisuje ją do sygnałów CNTx wyświetlanych na
-- LCD 
process(CLK)
    variable VAL0         : STD_LOGIC_VECTOR(3 downto 0) := "0000";
    variable VAL1         : STD_LOGIC_VECTOR(3 downto 0) := "0000";
    variable VAL2         : STD_LOGIC_VECTOR(3 downto 0) := "0000";
    variable CNT_VAL_PREV : STD_LOGIC_VECTOR(1 downto 0) := "00";
begin
    if rising_edge(CLK) then
            -- Warunki dla zwiększenia wartości licznika
        if      ((CNT_VAL_PREV & CNT_VAL) = "0001") or
                ((CNT_VAL_PREV & CNT_VAL) = "0110") or
                ((CNT_VAL_PREV & CNT_VAL) = "1011") or
                ((CNT_VAL_PREV & CNT_VAL) = "1100") then
            VAL0 := VAL0 + "0001";
            if VAL0 = "1010" then
                VAL0 := "0000";
                VAL1 := VAL1 + "0001";
                if VAL1 = "1010" then
                    VAL1 := "0000";
                    VAL2 := VAL2 + "0001";
                    if VAL2 = "1010" then
                        VAL2 := "0000";
                    end if;
                end if;
            end if;
            -- Warunki dla zmniejszenia wartości licznika
        elsif   ((CNT_VAL_PREV & CNT_VAL) = "1110") or
                ((CNT_VAL_PREV & CNT_VAL) = "1001") or
                ((CNT_VAL_PREV & CNT_VAL) = "0100") or
                ((CNT_VAL_PREV & CNT_VAL) = "0011") then
            VAL0 := VAL0 - "0001";
            if VAL0 = "1111" then
                VAL0 := "1001";
                VAL1 := VAL1 - "0001";
                if VAL1 = "1111" then
                    VAL1 := "1001";
                    VAL2 := VAL2 - "0001";
                    if VAL2 = "1111" then
                        VAL2 := "1001";
                    end if;
                end if;
            end if;
        end if;
        CNT0 <= VAL0;
        CNT1 <= VAL1;
        CNT2 <= VAL2;
        CNT_VAL_PREV := CNT_VAL;
    end if;
end process;

-- Preskaler (CLK_LOW na 1Hz)
process(CLK)
    variable PRESC : integer range 0 to 25000000;
begin
    if rising_edge(CLK) then
        PRESC := PRESC + 1;
        if PRESC = 0 then
            CLK_LOW <= not CLK_LOW;
        end if;
    end if;
end process;









-- Licznik mod 4
process(CLK)
    variable CLK_LOW_PREV   : STD_LOGIC;
    variable CNT4_VAL       : STD_LOGIC_VECTOR(1 downto 0);
begin
    if rising_edge(CLK) then
        if (CLK_LOW_PREV & CLK_LOW) = "01" then
            if SWITCH(0) = '1' then
                CNT4_VAL := CNT_VAL + '1';
            elsif SWITCH(0) = '0' then
                CNT4_VAL := CNT_VAL - '1';
            end if;
        end if;
        CLK_LOW_PREV := CLK_LOW;
        CNT_VAL <= CNT4_VAL;
    end if;
end process;

-- Algorytm 1
-- Enkoder w trakcie obrotu w jedną stronę generuje dwie fale prostokątne
-- przesunięte w fazie o ćwierć okresu. Zatem gdy na jednej z linii występuje
-- dowolne zbocze sprawdzając stan drugiej, jesteśmy w stanie rozpoznać kierunek
-- obrotu osi enkodera
process(CLK)
    variable ENC_A_PREV : STD_LOGIC;
    variable ENC_VAL    : STD_LOGIC_VECTOR(1 downto 0);
begin
    if rising_edge(CLK) then
        if (ENC_A_PREV & ENC_A) = "01" then
            if ENC_B = '1' then
                ENC_VAL := ENC_VAL + 1;
            else
                ENC_VAL := ENC_VAL - 1;
            end if;
        end if;
        CNT_VAL <= ENC_VAL;
    end if;
end process;

-- Algorytm 2
-- Algorytm ten polega na wstępnym debouncingu sygnałów linii enkodera
-- opierającego się na tym że w związku z jego budową wiemy, iż gdy
-- występuje zmiana na jednej z linii, druga linia jest stabilna.
-- Na odfiltrowanych sygnałach wykonywany jest Algorytm 1
process(CLK)
    variable ENC_A_LAST_PREV    : STD_LOGIC;
    variable ENC_A_LAST         : STD_LOGIC;
    variable ENC_B_LAST         : STD_LOGIC;
    variable ENC_A_PREV         : STD_LOGIC;
    variable ENC_B_PREV         : STD_LOGIC;
    variable ENC_VAL            : STD_LOGIC_VECTOR(1 downto 0);
begin
    if rising_edge(CLK) then
        if (ENC_A_PREV & ENC_A) = "01" then
            ENC_B_LAST := ENC_B;
        end if;

        if (ENC_B_PREV & ENC_B) = "01" then
            ENC_A_LAST := ENC_A;
        end if;

        if (ENC_A_LAST_PREV & ENC_A_LAST) = "01" then
            if ENC_B_LAST = '1' then
                ENC_VAL <= ENC_VAL + '1';
            elsif ENC_B_LAST = '0' then
                ENC_VAL <= ENC_VAL - '1';
            end if;
        end if;
        ENC_A_PREV := ENC_A;
        ENC_B_PREV := ENC_B;
        ENC_A_LAST_PREV := ENC_A_LAST;
        CNT_VAL <= ENC_VAL;
    end if;
end process;

-- Algorytm 3
-- Algorytm ten reaguje tylko na kilka konkretnych rodzajów zmian sygnałów
-- linii enkodera, eliminując reakcję na przejścia powstające wskutek
-- drgań zestyków
process(CLK)
    variable ENC_A_PREV : STD_LOGIC;
    variable ENC_B_PREV : STD_LOGIC;
    variable ENC_VAL    : STD_LOGIC_VECTOR(3 downto 0);
begin
    if rising_edge(CLK) then
        if      ((ENC_A & ENC_B) = "10" and (ENC_A_PREV & ENC_B_PREV) = "00") or
                ((ENC_A & ENC_B) = "11" and (ENC_A_PREV & ENC_B_PREV) = "10") or
                ((ENC_A & ENC_B) = "01" and (ENC_A_PREV & ENC_B_PREV) = "11") or
                ((ENC_A & ENC_B) = "00" and (ENC_A_PREV & ENC_B_PREV) = "01") then
            ENC_VAL <= ENC_VAL + '1';
        elsif   ((ENC_A & ENC_B) = "01" and (ENC_A_PREV & ENC_B_PREV) = "00") or
                ((ENC_A & ENC_B) = "11" and (ENC_A_PREV & ENC_B_PREV) = "01") or
                ((ENC_A & ENC_B) = "10" and (ENC_A_PREV & ENC_B_PREV) = "11") or
                ((ENC_A & ENC_B) = "00" and (ENC_A_PREV & ENC_B_PREV) = "10") then
            ENC_VAL <= ENC_VAL - '1';
        end if;
        ENC_A_PREV := ENC_A;
        ENC_B_PREV := ENC_B;
        CNT_VAL <= ENC_VAL(3 downto 2);
    end if;
end process;
\end{lstlisting}

\end{document}