%----------------------------------------------------------------------------------------
%	PACKAGES AND OTHER DOCUMENT CONFIGURATIONS
%----------------------------------------------------------------------------------------

\documentclass[12pt]{article}

\usepackage{polski}
\usepackage[polish]{babel}
\usepackage[utf8]{inputenc}
\usepackage{datetime}
\usepackage{graphicx}
\usepackage{tikz}
\usepackage{amsmath}
\usepackage{multirow}
\usepackage{tabularx}
\usepackage{geometry}
\geometry{
 	a4paper, 
 	left    = 20mm,
 	right	  = 20mm,
 	top     = 20mm,
 	bottom  = 20mm,
}
 
%----------------------------------------------------------------------------------------
 
%----------------------------------------------------------------------------------------
% DATES
%----------------------------------------------------------------------------------------

\renewcommand{\dateseparator}{.}
\newdate{exercise_date}{21}{10}{2014}

%----------------------------------------------------------------------------------------

%----------------------------------------------------------------------------------------
% TIKZ PACKAGES
%----------------------------------------------------------------------------------------

\usetikzlibrary{arrows}

%----------------------------------------------------------------------------------------

\begin{document}
 
\begin{titlepage}

\newcommand{\HRule}{\rule{\linewidth}{0.5mm}}
% Defines a new command for the horizontal lines, change thickness here

\center
% Center everything on the page
 
%----------------------------------------------------------------------------------------
%	LOGO SECTION
%----------------------------------------------------------------------------------------

\includegraphics[width=6cm]{../res/img/logo.png}\\[1cm]
% Include a department/university logo - this will require the graphicx package
 
%----------------------------------------------------------------------------------------
 
%----------------------------------------------------------------------------------------
%	HEADING SECTIONS
%----------------------------------------------------------------------------------------

\textsc{\LARGE Akademia Górniczo-Hutnicza \\[0.2cm]
im. Stanisława Staszica w Krakowie}\\[1.5cm]
% Name of your university/college

\textrm{\Large Wydział Elektrotechniki Automatyki Informatyki i Inżynierii
Biomedycznej}\\[1cm]

\textsc{\Large Laboratorium Aparatury Automatyzacji}\\[0.5cm]
% Major heading such as course name

%----------------------------------------------------------------------------------------
%	TITLE SECTION
%----------------------------------------------------------------------------------------

\HRule \\[0.4cm]
{ \huge \bfseries Prosty regulator mikroprocesorowy 
}\\%[0.4cm]
% Title of your document
\HRule \\[1.5cm]

%----------------------------------------------------------------------------------------
%	REPORT TABLE
%----------------------------------------------------------------------------------------

\begin{table}[h]
\centering
\begin{tabularx}{\linewidth}{|c|l|X|}
\hline
% \multicolumn{3}{|c|}{
% \begin{tabular}{cc}
% \begin{tabular}{c}
% \includegraphics[height=2.2cm]{../res/img/logo.jpg}\\
% \end{tabular}
% &
% \begin{tabular}{c}
% \Large{Akademia Górniczo-Hutnicza im. Stanisława Staszica}\\[5pt]
% \large{\textsc{Katedra Automatyki}}\\[5pt]
% \textsc{Laboratorium Aparatury Automatyzacji}
% \end{tabular}
% \end{tabular}
% }\\
% \hline
% \multicolumn{3}{|c|}{}\\[-5pt]
% \multicolumn{3}{|c|}{\textbf{\huge{Ćwiczenie 6}}}\\[10pt]
% \multicolumn{3}{|c|}{\Large{Bezpośrednie sterowanie cyfrowe}}\\[8pt]
% \hline
\multicolumn{2}{|c|}{Wydział EAIiIB, kierunek AiR, rok II}
& Grupa 6, wtorek 11:00-12:30\\
\hline
Lp. & Imię i nazwisko & Zaliczenie\\
\hline
1 & \textbf{Konrad Adasiewicz} & \\
\hline
2 & \textbf{Michał Maciejewski} & \\
\hline
3 & \textbf{Bartosz Szmit} & \\
\hline
\multicolumn{2}{|c|}{Data wykonania ćwiczenia:
\ddmmyyyydate\displaydate{exercise_date}r.} & Data oddania sprawozdania:
\ddmmyyyydate\displaydate{create_date}r.
\\
\hline
\end{tabularx}
\end{table}

%----------------------------------------------------------------------------------------

\vfill % Fill the rest of the page with whitespace

\end{titlepage} 

  \begin{section}{Zadanie 1}
    Dany jest układ o transmitancji przedstawionej na równaniu \ref{equ:g1}.
    
    \begin{equation}
      G_1(s)=\dfrac{k_1}{Ts+1}
      \label{equ:g1}
    \end{equation}
    \vspace{0.2cm}
    
    Jest on pobudzany szumem białym Gaussa o wariancji $\sigma^{2} = 1$. Na
    rysunku \ref{plot:x1} przedstawiona jest odpowiedź układu na takie
    wymuszenie.
    
    \begin{figure}[!htb]
      \begin{center}
        \includegraphics[width=14cm,trim=3cm 9cm 3cm 9cm,clip]
        {../res/img/z1_x.pdf}
      \end{center}
      \caption{Odpowiedź obiektu na wymuszenie w postaci białego szumu Gaussa}
      \label{plot:x1}
    \end{figure}
    
    Estymacja parametrów obiektu polega na dopasowaniu krzywej teoretycznego widma
    mocy sygnału do widma estymowanego na podstawie pomiaru. Teoretyczny
    przebieg widma mocy tego sygnału dany jest równaniem \ref{equ:ps}.
    
    \begin{equation}
      S(\omega)=|G(j\omega)|^2
      \label{equ:ps}
    \end{equation}
    \vspace{0.2cm}
    
    Co w przypadku systemu o transmitancji $G_1(s)$ oznacza widmo dane równaniem
    \ref{equ:ps1}.
    
    \begin{equation}
      S_1(\omega)=\dfrac{k_1^2}{T^2\omega^2+1}
      \label{equ:ps1}
    \end{equation}
    \vspace{0.2cm}
    
    \newpage
    
    Estymata widma mocy została wyznaczona metodą Yule'a-Walker'a. Przebiegi
    widma uzyskanego przy pomocy estymatora oraz jego aproksymacji krzywą daną równaniem
    \ref{equ:ps1} znajdują się na rysunku \ref{plot:approx1}.
    
    \begin{figure}[!htb]
      \begin{center}
        \includegraphics[width=14cm,trim=3cm 9cm 3cm 9cm,clip]
        {../res/img/z1_approx.pdf}
      \end{center}
      \caption{Estymata widma mocy oraz jej aproksymacja}
      \label{plot:approx1}
    \end{figure}
    
    Parametr $k_1$ obiektu został wyznaczony na podstawie tożsamości
    \ref{equ:ps1_0}.
    
    \begin{equation}
      S_1(0)=k_1^2
      \label{equ:ps1_0}
    \end{equation} 
    \vspace{0.2cm}
    
    Natomiast parametr $T$ został wyznaczony przy użyciu funkcji
    \textit{fminsearch} pakietu \textrm{MATLAB}
    
    Fitowanie krzywej pozwoliło wyznaczyć następujące parametry obiektu:
    \begin{itemize}
      \item $k_1 = 0.0917 [-]$
      \item $T = 1.211 [s]$
    \end{itemize}
  \end{section}
  
  \newpage

  \begin{section}{Zadanie 2}
    Dany jest układ o transmitancji:
    
    \begin{equation}
      G_1(s)=\dfrac{k_2}{s^2+2\xi\omega_0s+\omega_0^2}
      \label{equ:g2}
    \end{equation}
    \vspace{0.2cm}
    
    Jest on pobudzany szumem białym Gaussa o wariancji $\sigma^{2} = 1$. Na
    rysunku \ref{plot:x2} dana jest odpowiedź układu na takie wymuszenie.
    
    \begin{figure}[!htb]
      \begin{center}
        \includegraphics[width=14cm,trim=3cm 9cm 3cm 9cm,clip]
        {../res/img/z2_x.pdf}
      \end{center}
      \caption{Odpowiedź obiektu na wymuszenie w postaci białego szumu Gaussa}
      \label{plot:x2}
    \end{figure}
    
    Podobnie jak w poprzednim zadaniu należy dopasować krzywą estymowanego widma
    sygnału do jego teoretycznego odpowiednika danego równaniem \ref{equ:ps2}.
    
    \begin{equation}
      S_2(\omega)=\dfrac{k_2^2}{4\xi^2\omega_0^2\omega^2+(\omega_0^2-\omega^2)^2}
      \label{equ:ps2}
    \end{equation}
    \vspace{0.2cm}
    
    \newpage
    
    Estymata widma mocy została wyznaczona podobnie jak w zadaniu 1 metodą
    Yule'a-Walker'a. Przebiegi widma uzyskanego przy pomocy estymatora oraz
    jego aproksymacji krzywą daną równaniem \ref{equ:ps2} znajdują się na
    rysunku \ref{plot:approx2}.
    
    \begin{figure}[!htb]
      \begin{center}
        \includegraphics[width=14cm,trim=3cm 9cm 3cm 9cm,clip]
        {../res/img/z2_approx.pdf}
      \end{center}
      \caption{Estymata widma mocy oraz jej aproksymacja}
      \label{plot:approx2}
    \end{figure}
    
    Parametry krzywej aproksymacji widma zostały wyznaczone przy użyciu funkcji
    \textit{fminsearch} pakietu \textrm{MATLAB}.
    
    Fitowanie krzywej pozwoliło wyznaczyć następujące parametry obiektu:
    \begin{itemize}
      \item $k_2 = 1.084 [-]$
      \item $\xi = 0.0353 [-]$
      \item $\omega_0 = 1549.3 \left[\frac{1}{s}\right]$ ($246.5 [Hz]$)
    \end{itemize}

  \end{section}

  \begin{section}{Wnioski}
  
    Transmitancja operatorowa, a właściwie związana z nią transmitancja widmowa,
    opisuje dynamiczne właściwości obiektów w dziedzinie częstotliwości.
    Analizując zmianę widma na wyjściu badanego obiektu jesteśmy w stanie,
    znając charakter obiektu wyznaczyć jego parametry.
    
  \end{section}

\end{document}