%----------------------------------------------------------------------------------------
%	PACKAGES AND OTHER DOCUMENT CONFIGURATIONS
%----------------------------------------------------------------------------------------

\documentclass[12pt]{article}

\usepackage{polski}
\usepackage[polish]{babel}
\usepackage[utf8]{inputenc}
\usepackage{datetime}
\usepackage{graphicx}
\usepackage{tikz}
\usepackage{amsmath}
\usepackage{multirow}
\usepackage{tabularx}
\usepackage{geometry}
\geometry{
 	a4paper, 
 	left    = 20mm,
 	right	  = 20mm,
 	top     = 20mm,
 	bottom  = 20mm,
}
 
%----------------------------------------------------------------------------------------
 
%----------------------------------------------------------------------------------------
% DATES
%----------------------------------------------------------------------------------------

\renewcommand{\dateseparator}{.}
\newdate{exercise_date}{21}{10}{2014}

%----------------------------------------------------------------------------------------

%----------------------------------------------------------------------------------------
% TIKZ PACKAGES
%----------------------------------------------------------------------------------------

\usetikzlibrary{arrows}

%----------------------------------------------------------------------------------------

\begin{document}
 
\begin{titlepage}

\newcommand{\HRule}{\rule{\linewidth}{0.5mm}}
% Defines a new command for the horizontal lines, change thickness here

\center
% Center everything on the page
 
%----------------------------------------------------------------------------------------
%	LOGO SECTION
%----------------------------------------------------------------------------------------

\includegraphics[width=6cm]{../res/img/logo.png}\\[1cm]
% Include a department/university logo - this will require the graphicx package
 
%----------------------------------------------------------------------------------------
 
%----------------------------------------------------------------------------------------
%	HEADING SECTIONS
%----------------------------------------------------------------------------------------

\textsc{\LARGE Akademia Górniczo-Hutnicza \\[0.2cm]
im. Stanisława Staszica w Krakowie}\\[1.5cm]
% Name of your university/college

\textrm{\Large Wydział Elektrotechniki Automatyki Informatyki i Inżynierii
Biomedycznej}\\[1cm]

\textsc{\Large Laboratorium Aparatury Automatyzacji}\\[0.5cm]
% Major heading such as course name

%----------------------------------------------------------------------------------------
%	TITLE SECTION
%----------------------------------------------------------------------------------------

\HRule \\[0.4cm]
{ \huge \bfseries Prosty regulator mikroprocesorowy 
}\\%[0.4cm]
% Title of your document
\HRule \\[1.5cm]

%----------------------------------------------------------------------------------------
%	REPORT TABLE
%----------------------------------------------------------------------------------------

\begin{table}[h]
\centering
\begin{tabularx}{\linewidth}{|c|l|X|}
\hline
% \multicolumn{3}{|c|}{
% \begin{tabular}{cc}
% \begin{tabular}{c}
% \includegraphics[height=2.2cm]{../res/img/logo.jpg}\\
% \end{tabular}
% &
% \begin{tabular}{c}
% \Large{Akademia Górniczo-Hutnicza im. Stanisława Staszica}\\[5pt]
% \large{\textsc{Katedra Automatyki}}\\[5pt]
% \textsc{Laboratorium Aparatury Automatyzacji}
% \end{tabular}
% \end{tabular}
% }\\
% \hline
% \multicolumn{3}{|c|}{}\\[-5pt]
% \multicolumn{3}{|c|}{\textbf{\huge{Ćwiczenie 6}}}\\[10pt]
% \multicolumn{3}{|c|}{\Large{Bezpośrednie sterowanie cyfrowe}}\\[8pt]
% \hline
\multicolumn{2}{|c|}{Wydział EAIiIB, kierunek AiR, rok II}
& Grupa 6, wtorek 11:00-12:30\\
\hline
Lp. & Imię i nazwisko & Zaliczenie\\
\hline
1 & \textbf{Konrad Adasiewicz} & \\
\hline
2 & \textbf{Michał Maciejewski} & \\
\hline
3 & \textbf{Bartosz Szmit} & \\
\hline
\multicolumn{2}{|c|}{Data wykonania ćwiczenia:
\ddmmyyyydate\displaydate{exercise_date}r.} & Data oddania sprawozdania:
\ddmmyyyydate\displaydate{create_date}r.
\\
\hline
\end{tabularx}
\end{table}

%----------------------------------------------------------------------------------------

\vfill % Fill the rest of the page with whitespace

\end{titlepage} 

  \begin{section}{Zadanie 1}
    Dany jest układ \textrm{ARMAX} o strukturze z równania \ref{equ:armax1}.
    
    \begin{equation}
      y_n=-a_1y_{n-1}+b_1u_{n-1}+v_i 
      \label{equ:armax1}
    \end{equation}
    \vspace{0.2cm}
    
    Jest on pobudzany sygnałem prostokątnym o amplitudzie $4$, natomiast sygnał
    $v$ jest szumem zakłóceń. Na rysunku \ref{plot:approx1} znajduje się
    odpowiedź wyznaczonego modelu oraz w tle przebieg odpowiedzi zmierzonej, na
    podstawie której wyznaczono parametry modelu.
    
    \begin{figure}[!htb]
      \begin{center}
        \includegraphics[width=14cm,trim=3cm 9cm 3cm 9cm,clip]
        {../res/img/z1_approx.pdf}
      \end{center}
      \caption{Odpowiedź wyznaczonego modelu \ref{equ:armax1} na tle dokonanych pomiarów}
      \label{plot:approx1}
    \end{figure}
    
    Parametry modelu zostały znalezione metodą najmniejszych kwadratów.
    Wyznaczenie parametrów polega na wykorzystaniu tożsamości podanej w równaniu
    \ref{equ:lsq}. Tożsamość ta wywodzi się z zapisania uogólnionej wersji
    równania \ref{equ:armax1} dla wszystkich $N$ posiadanych próbek
    pomiarowych, z pominięciem szumu $v$. $k$ oznacza rząd modelu.
    
    \begin{equation}
      \begin{bmatrix}
        y_{k+1} \\ y_{k+2} \\ \vdots \\ y_N
      \end{bmatrix} =
      \begin{bmatrix}
        y_{k} & y_{k-1} & \hdots & y_{1} & u_{k} & u_{k-1} & \hdots & u_1 \\
        y_{k+1} & y_{k} & \hdots & y_{2} & u_{k+1} & u_{k} & \hdots & u_2 \\
        \vdots  & \vdots  & \vdots & \vdots & \vdots & \vdots  &\vdots  & \vdots \\
        y_{N-1} & y_{N-2} & \hdots & y_{N-k} & u_{N-1} & u_{N-2} & \hdots &
        u_{N-k} \\
      \end{bmatrix}
      \begin{bmatrix}
        a_1 \\ a_2 \\ \vdots \\ a_k \\ b_1 \\ b_2 \\ \vdots \\ b_k 
      \end{bmatrix}
      \label{equ:lsq}
    \end{equation}
    \vspace{0.2cm}
    
    \newpage
    
    Równanie \ref{equ:lsq} w wersji skróconej można zapisać w postaci
    \ref{equ:lsqs}.
    
    \begin{equation}
      Y = \Phi C
      \label{equ:lsqs} 
    \end{equation}
    \vspace{0.2cm}
    
    Po prostych przekształceniach dochodzimy do równania \ref{equ:clsqs} pozwalającego
    wyznaczyć wektor współczynników $C$.
    
    \begin{equation}
      C = (\Phi^T\Phi)^{-1}\Phi^T Y
      \label{equ:clsqs} 
    \end{equation}
    \vspace{0.2cm}
    
    Oszacowanie błędu parametrów polega na obliczeniu macierzy kowariancji
    współczynników, do której wyznaczenia użyto równania \ref{equ:cove}
    opierającego się o zasadę propagacji błędów statystycznych.
    $\sigma^2$ jest wariancją różnicy sygnałów wyjściowych zmierzonego oraz z
    wyznaczonego modelu.
    
    \begin{equation}
      \mathrm{Cov}(C) = \sigma^2(\Phi^T\Phi)^{-1}
      \label{equ:cove} 
    \end{equation}
    \vspace{0.2cm}
    
    Wyznaczone parametry to:
    
    \begin{itemize}
      \item $a_1 = -0.9161$, $\sigma_{a_1} = 0.0097$
      \item $b_1 = 0.0581$, $\sigma_{b_1} = 0.0075$
    \end{itemize}
    
    Drugą częścią zadania było wyznaczenie parametrów metodą predykcji błędów,
    co sprowadziło zadanie do użycia funkcji \textit{armax} pakietu
    \textrm{MATLAB}. Funkcja przyjmuje jako argumenty zmierzone sygnały, oraz
    rzędy wielomianów modelu \textrm{ARMAX}. Wyznaczone przy jej pomocy
    parametry są identyczne (różnice na poziomie $10^{-14}$) jak te wyznaczone
    metodą najmniejszych kwadratów, jednak dokładność ich wyznaczenia jest wyższa:
    
    \begin{itemize}
      \item $a_1 = -0.9161$, $\sigma_{a_1} = 0.0036$
      \item $b_1 = 0.0581$, $\sigma_{b_1} = 0.0028$
    \end{itemize}
    
    \newpage
    
    Ostatnim zadaniem było sprawdzenie czy reszty modelu \ref{equ:armax1} są
    białym szumem. W tym celu wyznaczono transformatę Fouriera reszt, która
    przedstawiona jest na rysunku \ref{plot:errfft1}.
    
    \begin{figure}[!htb]
      \begin{center}
        \includegraphics[width=14cm,trim=3cm 9cm 3cm 9cm,clip]
        {../res/img/z1_errfft.pdf}
      \end{center}
      \caption{Widmo reszt modelu \ref{equ:armax1}}
      \label{plot:errfft1}
    \end{figure}
    
    Jak zatem widać otrzymane reszty nie są szumem białym.
  
    \newpage
    
  \end{section}

  \begin{section}{Zadanie 2}
    Dany jest układ \textrm{ARMAX} o strukturze z równania \ref{equ:armax2}.
    
    \begin{equation}
      y_n=-a_1y_{n-1}-a_2y_{n-2}+b_1u_{n-1}+b_2u_{n-2}+v_i 
      \label{equ:armax2}
    \end{equation}
    \vspace{0.2cm}
    
    Tak jak w zadaniu 1 jest on pobudzany sygnałem prostokątnym o amplitudzie
    $4$, natomiast sygnał $v$ jest szumem zakłóceń. Na rysunku
    \ref{plot:approx2} znajduje się odpowiedź wyznaczonego modelu oraz w tle
    przebieg odpowiedzi zmierzonej, na podstawie której wyznaczono parametry
    modelu.
    
    \begin{figure}[!htb]
      \begin{center}
        \includegraphics[width=14cm,trim=3cm 9cm 3cm 9cm,clip]
        {../res/img/z2_approx.pdf}
      \end{center}
      \caption{Odpowiedź wyznaczonego modelu \ref{equ:armax2} na tle dokonanych pomiarów}
      \label{plot:approx2}
    \end{figure}
    
    \newpage
    
    Parametry modelu zostały wyznaczone analogicznie jak w zadaniu 1 zatem
    podane zostaną tylko wyniki.
    
    Wyznaczone parametry to:
    
    \begin{itemize}
      \item $a_1 = -1.5523$, $\sigma_{a_1} = 0.0124$
      \item $a_2 = 0.9037$, $\sigma_{a_2} = 0.0116$
      \item $b_1 = 0.1823$, $\sigma_{b_1} = 0.0199$
      \item $b_2 = 0.1694$, $\sigma_{b_2} = 0.0212$
    \end{itemize}
    
    Dla metody predykcji błędu, znów wartości parametrów okazały się być takie
    same, jednak ta metoda cechuje się ponownie niższym oszacowaniem błędów
    wyznaczonych współczynników.
    
    \begin{itemize}
      \item $a_1 = -1.5523$, $\sigma_{a_1} = 0.0027$
      \item $a_2 = 0.9037$, $\sigma_{a_2} = 0.0026$
      \item $b_1 = 0.1823$, $\sigma_{b_1} = 0.0044$
      \item $b_2 = 0.1694$, $\sigma_{b_2} = 0.0047$
    \end{itemize}
    
    Widmo reszt modelu pokazane na rysunku \ref{plot:errfft2} wskazuje na to iż
    ponownie nie są one szumem białym.
    
    \begin{figure}[!htb]
      \begin{center}
        \includegraphics[width=14cm,trim=3cm 9cm 3cm 9cm,clip]
        {../res/img/z2_errfft.pdf}
      \end{center}
      \caption{Widmo reszt modelu \ref{equ:armax2}}
      \label{plot:errfft2}
    \end{figure}
    
  \end{section}

  \begin{section}{Wnioski}
  
    Identyfikację procesów warto przeprowadzać w oparciu o dwie niezależne
    metody. Pozwala to uniknąć błędów ludzkich, ponieważ znaczna rozbieżność
    wartości parametrów wyznaczonych różnymi metodami będzie wskazywać na
    popełnienie błędu w procesie identyfikacji w jednej, kilku, lub wszystkich
    użytych metodach.
    
    Metoda najmniejszych kwadratów pomimo upływu czasu dalej jest dobrym
    estymatorem parametrów. Widać to na przykładzie wyznaczonych współczynników
    badanych modeli.
    
  \end{section}

\end{document}