

%----------------------------------------------------------------------------------------
%	PACKAGES AND OTHER DOCUMENT CONFIGURATIONS
%----------------------------------------------------------------------------------------

\documentclass[12pt]{article}

\usepackage{polski}
\usepackage[polish]{babel}
\usepackage[utf8]{inputenc} 
\usepackage{datetime}
\usepackage{graphicx}
\usepackage{tikz} 
\usepackage{amsmath}
\usepackage{epstopdf}
\usepackage{multirow}
\usepackage{tabularx}
%\usepackage[colorlinks=true]{hyperref}
%\usepackage[all]{hypcap}
%\usepackage{showframe} 
\usepackage{geometry}
 \geometry{
 a4paper, 
 left=20mm,
 right=20mm,
 top=20mm,
 bottom=20mm,
 }
 
\renewcommand{\dateseparator}{.}
\newdate{exercise_date}{29}{04}{2014}
\newdate{create_date}{06}{05}{2014}

%----------------------------------------------------------------------------------------

%----------------------------------------------------------------------------------------
% TIKZ PACKAGES
%----------------------------------------------------------------------------------------
 
\usetikzlibrary{arrows, decorations.markings, decorations.pathmorphing, calc,
positioning, intersections, shapes, decorations.pathreplacing}

%----------------------------------------------------------------------------------------

\begin{document}

\begin{titlepage}

\newcommand{\HRule}{\rule{\linewidth}{0.5mm}}
% Defines a new command for the horizontal lines, change thickness here

\center
% Center everything on the page
 
%----------------------------------------------------------------------------------------
%	LOGO SECTION
%----------------------------------------------------------------------------------------

\includegraphics[width=6cm]{../res/img/logo.png}\\[1cm]
% Include a department/university logo - this will require the graphicx package
 
%----------------------------------------------------------------------------------------
 
%----------------------------------------------------------------------------------------
%	HEADING SECTIONS
%----------------------------------------------------------------------------------------

\textsc{\LARGE Akademia Górniczo-Hutnicza \\[0.2cm]
im. Stanisława Staszica w Krakowie}\\[1.5cm]
% Name of your university/college

\textrm{\Large Wydział Elektrotechniki Automatyki Informatyki i Inżynierii
Biomedycznej}\\[1cm]

\textsc{\Large Laboratorium Aparatury Automatyzacji}\\[0.5cm]
% Major heading such as course name

%----------------------------------------------------------------------------------------
%	TITLE SECTION
%----------------------------------------------------------------------------------------

\HRule \\[0.4cm]
{ \huge \bfseries Prosty regulator mikroprocesorowy 
}\\%[0.4cm]
% Title of your document
\HRule \\[1.5cm]

%----------------------------------------------------------------------------------------
%	REPORT TABLE
%----------------------------------------------------------------------------------------

\begin{table}[h]
\centering
\begin{tabularx}{\linewidth}{|c|l|X|}
\hline
% \multicolumn{3}{|c|}{
% \begin{tabular}{cc}
% \begin{tabular}{c}
% \includegraphics[height=2.2cm]{../res/img/logo.jpg}\\
% \end{tabular}
% &
% \begin{tabular}{c}
% \Large{Akademia Górniczo-Hutnicza im. Stanisława Staszica}\\[5pt]
% \large{\textsc{Katedra Automatyki}}\\[5pt]
% \textsc{Laboratorium Aparatury Automatyzacji}
% \end{tabular}
% \end{tabular}
% }\\
% \hline
% \multicolumn{3}{|c|}{}\\[-5pt]
% \multicolumn{3}{|c|}{\textbf{\huge{Ćwiczenie 6}}}\\[10pt]
% \multicolumn{3}{|c|}{\Large{Bezpośrednie sterowanie cyfrowe}}\\[8pt]
% \hline
\multicolumn{2}{|c|}{Wydział EAIiIB, kierunek AiR, rok II}
& Grupa 6, wtorek 11:00-12:30\\
\hline
Lp. & Imię i nazwisko & Zaliczenie\\
\hline
1 & \textbf{Konrad Adasiewicz} & \\
\hline
2 & \textbf{Michał Maciejewski} & \\
\hline
3 & \textbf{Bartosz Szmit} & \\
\hline
\multicolumn{2}{|c|}{Data wykonania ćwiczenia:
\ddmmyyyydate\displaydate{exercise_date}r.} & Data oddania sprawozdania:
\ddmmyyyydate\displaydate{create_date}r.
\\
\hline
\end{tabularx}
\end{table}

%----------------------------------------------------------------------------------------

\vfill % Fill the rest of the page with whitespace

\end{titlepage}

\section{Wstęp}

\subsection{Cel ćwiczenia}

Zapoznanie się z budową i programowaniem prostego jednokanałowego regulatora cyfrowego, 
stosowanego do regulacji temperatury lub innych wielkości przetworzonych na sygnał 
standardowy w niewielkich instalacjach (np. w zakładach rzemieślniczych). Zapoznanie się z 
przykładem jednoobwodowego układu regulacji temperatury.

\subsection{Opis stanowiska} 

Stanowisko składa się z dwóch części połączonych przewodem wielożyłowym z
wtyczkami:

\begin{itemize}
	\item Skrzynka z regulatorem. Zawiera regulator \textsc{LUMEL RE31} wraz z
  	połączeniami elektrycznymi oraz włącznik sieciowy. Zastosowano regulator z
  	wyjściem przekaźnikowym dwupołożeniowym $230[V]$ $3[A]$.
  	\item Obiekt regulacji typu cieplnego. Jest nim walec aluminiowy zamocowany we wkładzie 
	grzejnym do lutownicy o mocy $400[W]$. Spirala grzejna jest zasilana impulsowo
	z wyjścia regulatora sygnałem sieciowym $230[V]$. Współczynnik wypełnienia
	sygnału sterującego jest modulowany poziomem sygnału wyjściowego regulatora.
	Na obu końcach walca są symetrycznie zamocowane dwa termometry \textsc{Pt100}.
	Jeden z nich jest połączony z wejściem regulatora, a drugi z gniazdami na
	przedniej ścianie obudowy. Oba termometry są łączone w systemie
	trójprzewodowym. Wystające z wkładu końce walca są karbowane w celu
	zwiększenia intensywności chłodzenia.
	Obiekt umieszczony jest w metalowej obudowie. W prawej ścianie obudowy jest
	wbudowany wentylator uruchamiany z wyjścia alarmowego \textsc{AL1} regulatora
	stosowany w celu dodatkowego chłodzenia obiektu.
\end{itemize}

\begin{figure}[!htb]
	\begin{center}
		\tikzstyle{vecArrow} = 
	[	thick,
		decoration={
			markings,
			mark=at position 0+20pt with
				{\arrowreversed[thin, scale=2]{open triangle 60}},
			mark=at position 1 with {\arrow[thin, scale=2]{open triangle 60}}
			},
		double distance=4pt,
		shorten >= 10pt,
		shorten <= 10pt,
		preaction = {decorate},
		postaction = {
		draw,
		line width=4pt,
		white,
		shorten >= 9pt,
		shorten <= 9pt
		}
  ]
  
\tikzstyle{vecArrowSS} = 
	[	thick,
		decoration={
			markings,
			mark=at position 1 with
				{\arrow[thin, scale=2]{open triangle 60}}
			},
		double distance=4pt,
		shorten >= 10pt,
		preaction = {decorate},
		postaction = {
		draw,
		line width=4pt,
		white,
		shorten >= 9pt,
		}
  ]
  
\tikzstyle{innerWhite} = 
	[	semithick,
		white,
		line width=4pt,
		shorten >= 9pt,
		shorten <= 9pt
	]

\begin{tikzpicture}[
            % copy to preamble to use it everywhere \tikzset{node/.style...}
            node/.style=
            {
                % shape
                rectangle,
                rounded corners=1mm,
                minimum size=12mm,
                %border
                thick,
                draw,
            },
            busnode/.style=
            {
                inner sep=0,
                minimum size=0mm,
            },
            circ/.style=
            {
                % shape
                circle,
                minimum size=12mm,
                %border
                thick,
                draw,
            },
            node distance=10mm
            ]
	\node[node]
				(PC)
				{$\textsc{PC}$};
	\node[node,
				right=30mm of PC]
				(ADAM5000)
				{$\textsc{ADAM5000}$};
 	\node[busnode,
 				below=15mm of ADAM5000]
 				(bus60)
 				{};
	\node[node,
				right=of bus60]
				(ADAM5060)
				{$\textsc{ADAM5060}$};
	\node[circ,
				right=of ADAM5060]
				(Lamp)
				{};
	\draw[thick]
				(Lamp.north west) -- (Lamp.south east);
	\draw[thick]
				(Lamp.north east) -- (Lamp.south west);
				
	\node[busnode,
				below=20mm of bus60]
				(bus18)
				{};
	\node[node,
				right=of bus18]
				(ADAM5018)
				{$\textsc{ADAM5018}$};
	\node[circ,
				right=of ADAM5018]
				(Fotocop)
				{};
	\draw[thick]
				(Fotocop.south) -- +(0,0.5);
	\draw[thick]
				(Fotocop.north) -- +(0,-0.5);
	\draw[thick]
				(Fotocop.north) +(-0.4,-0.5)	--
				+(0.4,-0.5);
	\draw[ultra thick]
				(Fotocop.south) +(-0.2,0.5)	--
				+(0.2,0.5);
	\draw[-latex, thick]
				(Fotocop.east)
				+(0.5,0.3) --
				+(0.1,0.0);
	\draw[-latex, thick]
				(Fotocop.east)
				+(0.5,0.1) --
				+(0.1,-0.2);
				
	\node[busnode,
				below=20mm of bus18]
				(bus13)
				{};
	\node[node,
				right=of bus13]
				(ADAM5013)
				{$\textsc{ADAM5013}$};
	\node[node,
				right=of ADAM5013]
				(Pt100)
				{\text{\hspace{2cm}}};
	\draw
				(Pt100.west) +(0.5,0.0) -- +(0.7,0.0);
	\draw[decorate, decoration={zigzag}]
				(Pt100.west) +(0.7,0.0) --
				($(Pt100.east)-(0.72,0.0)$);
	\draw
				(Pt100.east) +(-0.52,0.0) -- +(-0.72,0.0);
	\node[below=1mm of Pt100]
				(Pt100name)
				{\textsc{Pt100}};

				
	\draw[vecArrow]
				(PC) to node[above]{RS-232} (ADAM5000);
	\draw[vecArrow]
				(ADAM5000) |- (ADAM5013);
	\draw[vecArrowSS]
				(bus18) to (ADAM5018);
	\draw[vecArrowSS]
				(bus60) to (ADAM5060);
	\draw[innerWhite]
				(ADAM5000) |- (ADAM5013);
				
	\draw[-latex,very thick]
				(ADAM5060) -- (Lamp);
	\draw[-latex,very thick]
				(Fotocop) -- (ADAM5018);
	\draw[-latex,very thick]
				(Pt100) -- (ADAM5013);
	
\end{tikzpicture}
	\end{center}
	\caption{Schemat stanowiska}
\end{figure}

\newpage

\subsection{Opis regulatora LUMEL RE31}

Regulator \textsc{LUMEL RE31} jest regulatorem cyfrowym, który umożliwia
implementację regulacji dwupołożeniowej, oraz PID z wyjściem PWM dla
wolnozmiennych procesów regulowanych. Menu programowania sterownika, podzielone
jest na 3 poziomy bezpieczeństwa, w zależności od częstości zmian danych
funkcji oraz konsekwencji, które mogą wyniknąć z ich przypadkowej zmiany.
Menu jest obsługiwane trzema przyciskami: góra, dół, akceptacja. Przyciskiem
akceptacji eksplorujemy menu, chcąc przejść do wyższego poziomu musimy dojść do
ostatniej opcji aktualnego poziomu i przytrzymać przycisk akceptacji przez
$3.2[s]$. Zagapienie się i ominięcie ostatniej opcji poziomu powoduje powrót do
najniższej opcji najniższego poziomu.

\subsection{Regulacja PWM}

Fundamentem regulacji sygnałem PWM jest idea kontroli przepływu energii
dostarczanej do obiektu małymi porcjami, z mocą maksymalną lub minimalną, dzięki
czemu sterowanie jej przepływem jest prostsze i odbywa się z mniejszymi
stratami. Przy regulacji PWM sygnałem regulującym jest sygnał prostokątny o
stałej amplitudzie i częstotliwości, o zmiennym współczynniku wypełnienia.

\begin{figure}[!htb]
	\begin{center}
		\begin{tikzpicture}[scale=2]
			\draw
				(0,0)	node[left]{$0$};
			\draw
				(0,2)	node[left]{$U_{max}$}; 
			
			\draw[-latex,ultra thick]
				(0,0)	--
				(5,0)	node[below]{$t[s]$};
			\draw[-latex,ultra thick]
				(0,0)	--
				(0,3)	node[left]{$u(t)[-]$};
			\draw
				(0,2)		--
				++(1.5,0)	--
				++(0,-2)
				++(0.5,0)	--
				++(0,2)		--
				++(1,0)		--
				++(0,-2)
				++(1,0)		--
				++(0,2)		--
				++(0.5,0);
				
			\draw[decorate,decoration={brace,mirror,amplitude=7pt}]
				(0,0)	-- node[below,yshift=-18]{$t_{w1}$}
				++(1.5,0);
			\draw[decorate,decoration={brace,mirror,amplitude=7pt}]
				(1.5,0)	-- node[below,yshift=-18]{$t_{n1}$}
				++(0.5,0);
			\draw[decorate,decoration={brace,mirror,amplitude=7pt}]
				(2,0)	-- node[below,yshift=-18]{$t_{w2}$}
				++(1,0);
			\draw[decorate,decoration={brace,mirror,amplitude=7pt}]
				(3,0)	-- node[below,yshift=-18]{$t_{n2}$}
				++(1,0);
				
		\end{tikzpicture}
	\end{center}
	\caption{Regulacja PWM}
\end{figure}

Średnia wartość sygnału dostarczana do obiektu jest równa
$U_{\text{śr}}=U_{max}\dfrac{t_w}{t_w+t_n}[-]$.

\newpage

\section{Przebieg ćwiczenia}

Przed rozpoczęciem ćwiczenia należało ustawić pewne parametry regulatora
charakterystyczne dla badanej konfiguracji układu regulacji:

\begin{table}[!htb]
	\begin{center}
		\begin{tabular}{|c|c|}
			\hline
			Typ wejścia	&	\textsc{Pt100 JS}	\\
			\hline
			Dolna granica zakresu 	&	$10^{\circ}C$	\\
			\hline
			Górna granica zakresu 	&	$300^{\circ}C$	\\
			\hline
			Dolne ograniczenie mocy wyjściowej 	&	$0\%$	\\
			\hline
			Górne ograniczenie mocy wyjściowej 	&	$100\%$	\\
			\hline
			Wartości zadane	&	$100^{\circ}C$, $200^{\circ}C$	\\
			\hline
			Alarm AL1 względny	&	$+2^{\circ}C$, $+10^{\circ}C$, $+50^{\circ}C$	\\
			\hline
			Alarm AL1 bezwzględny	&	Zadana + $2^{\circ}C$, $10^{\circ}C$,
			$50^{\circ}C$	\\
			\hline
			Rodaj regulacji	&	Inwersyjne	\\
			\hline
		\end{tabular}
	\end{center}
	\label{tab:params}
	\caption{Parametry regulatora}
\end{table}

Pierwszą częścią naszego ćwiczenia było zbadanie działania regulatora, jako regulatora 
dwustawnego dla różnych wartości histerezy. Aby uruchomić powyższy tryb należało 
ustawić wartość zakresu proporcjonalności $P_b$ na 0, oraz ustawić
porządaną wartość histerezy.

Drugą część ćwiczenia stanowiło 
sprawdzenie pracy regulatora w trybie PID, przy poniższych nastawach regulatora:

\begin{table}[!htb]
	\begin{center}
		\begin{tabular}{|c|c|c|}
			\hline
			Wartość zadana	&	$100^{\circ}C$	&	$200^{\circ}C$ \\
			\hline
			$P_b$	&	40	&	16 \\
			\hline
			$T_i$	&	166	&	103 \\
			\hline
			$T_d$	&	209	&	51 \\
			\hline
		\end{tabular}
	\end{center}
	\label{tab:nast}
	\caption{Nastawy regulatora PID}
\end{table}

\section{Wnioski i spostrzeżenia}

Sterowanym obiektem jest masywna bryła metalowa, jest to obiekt wysokiego rzędu
posiadający dużą bezwładność cieplną, co skutecznie utrudnia jego sterowanie.

Sterownik \textsc{LUMEL RE31} posiada wyjątkowo niewdzięczny interfejs
użytkownika, zapewne jest dedykowany do obsługi procesów, które nie wymagają
częstej zmiany nastaw regulatora.

W przypadku pracy regulatora w trybie dwupołożeniowym dało się zaobserwować duże
przeregulowania wartości sterowanej, dla wartości zadanej $100^{\circ}C$ i
histerezy jedynie $2^{\circ}C$ temperatura obiektu sięgała w szczycie nawet
$110^{\circ}C$. Dla wartości zadanej $200^{\circ}C$
przeregulowania względne pozostawały na w przybliżeniu takim samym poziomie.
Dla niektórych zastosowań takie przebiegi wielkości regulowanej są niedopuszczalne.

W przypadku pracy w trybie PID PWM, regulator działał znacznie sprawniej.
Dla wartości zadanej $100^{\circ}C$ pierwsze przeregulowanie na
poziomie $5^{\circ}C$, następnie po ustabilizowaniu wartość regulowana
oscylowała wokół wartości zadanej w zakresie mniejszym niż $\pm 1^{\circ}C$

Podsumowując, regulacja dwustawna jest bardzo prostym algorytmem regulacyjnym,
jednak w zastosowaniach gdzie niemile widziane są duże oscylacje wartości
regulowanej jest bezużyteczna. Pracę tego regulatora zaburzają opóźnienia
odpowiedzi obiektu, ponieważ wartość regulowana rośnie nawet pomimo już
wyłączonego wymuszenia, co powoduje znaczne oscylacje odpowiedzi obiektu.
Regulacja ciągła PID zapewnia dużo lepsze parametry regulacyjne.


\end{document}


