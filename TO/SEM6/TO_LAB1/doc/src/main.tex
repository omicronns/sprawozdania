%----------------------------------------------------------------------------------------
%	PACKAGES AND OTHER DOCUMENT CONFIGURATIONS
%----------------------------------------------------------------------------------------

\documentclass[12pt]{article}

\usepackage{polski}
\usepackage[polish]{babel}
\usepackage[utf8]{inputenc}
\usepackage{datetime}
\usepackage{graphicx}
\usepackage{tikz}
\usepackage{amsmath}
\usepackage{amsfonts}
\usepackage{epstopdf}
\usepackage{multirow}
\usepackage{tabularx}
%\usepackage[colorlinks=true]{hyperref}
%\usepackage[all]{hypcap}
%\usepackage{showframe} 
\usepackage{geometry}
 \geometry{
 	a4paper, 
 	left	=20mm,
 	right	=20mm,
 	top		=20mm,
 	bottom	=20mm,
 }
 
%----------------------------------------------------------------------------------------
 
%----------------------------------------------------------------------------------------
% DATES
%----------------------------------------------------------------------------------------

\renewcommand{\dateseparator}{.}
\newdate{exercise_date}{20}{05}{2015}

%----------------------------------------------------------------------------------------

%----------------------------------------------------------------------------------------
% TIKZ PACKAGES
%----------------------------------------------------------------------------------------

\usetikzlibrary{arrows}

%----------------------------------------------------------------------------------------

\begin{document}
 
\begin{titlepage}

\newcommand{\HRule}{\rule{\linewidth}{0.5mm}}
% Defines a new command for the horizontal lines, change thickness here

\center
% Center everything on the page
 
%----------------------------------------------------------------------------------------
%	LOGO SECTION
%----------------------------------------------------------------------------------------

\includegraphics[width=6cm]{../res/img/logo.png}\\[1cm]
% Include a department/university logo - this will require the graphicx package
 
%----------------------------------------------------------------------------------------
 
%----------------------------------------------------------------------------------------
%	HEADING SECTIONS
%----------------------------------------------------------------------------------------

\textsc{\LARGE Akademia Górniczo-Hutnicza \\[0.2cm]
im. Stanisława Staszica w Krakowie}\\[1.5cm]
% Name of your university/college

\textrm{\Large Wydział Elektrotechniki Automatyki Informatyki i Inżynierii
Biomedycznej}\\[1cm]

\textsc{\Large Laboratorium Aparatury Automatyzacji}\\[0.5cm]
% Major heading such as course name

%----------------------------------------------------------------------------------------
%	TITLE SECTION
%----------------------------------------------------------------------------------------

\HRule \\[0.4cm]
{ \huge \bfseries Prosty regulator mikroprocesorowy 
}\\%[0.4cm]
% Title of your document
\HRule \\[1.5cm]

%----------------------------------------------------------------------------------------
%	REPORT TABLE
%----------------------------------------------------------------------------------------

\begin{table}[h]
\centering
\begin{tabularx}{\linewidth}{|c|l|X|}
\hline
% \multicolumn{3}{|c|}{
% \begin{tabular}{cc}
% \begin{tabular}{c}
% \includegraphics[height=2.2cm]{../res/img/logo.jpg}\\
% \end{tabular}
% &
% \begin{tabular}{c}
% \Large{Akademia Górniczo-Hutnicza im. Stanisława Staszica}\\[5pt]
% \large{\textsc{Katedra Automatyki}}\\[5pt]
% \textsc{Laboratorium Aparatury Automatyzacji}
% \end{tabular}
% \end{tabular}
% }\\
% \hline
% \multicolumn{3}{|c|}{}\\[-5pt]
% \multicolumn{3}{|c|}{\textbf{\huge{Ćwiczenie 6}}}\\[10pt]
% \multicolumn{3}{|c|}{\Large{Bezpośrednie sterowanie cyfrowe}}\\[8pt]
% \hline
\multicolumn{2}{|c|}{Wydział EAIiIB, kierunek AiR, rok II}
& Grupa 6, wtorek 11:00-12:30\\
\hline
Lp. & Imię i nazwisko & Zaliczenie\\
\hline
1 & \textbf{Konrad Adasiewicz} & \\
\hline
2 & \textbf{Michał Maciejewski} & \\
\hline
3 & \textbf{Bartosz Szmit} & \\
\hline
\multicolumn{2}{|c|}{Data wykonania ćwiczenia:
\ddmmyyyydate\displaydate{exercise_date}r.} & Data oddania sprawozdania:
\ddmmyyyydate\displaydate{create_date}r.
\\
\hline
\end{tabularx}
\end{table}

%----------------------------------------------------------------------------------------

\vfill % Fill the rest of the page with whitespace

\end{titlepage}

\section*{Abstrakt}

Przed autorem został postawiony problem stworzenia aplikacji w pakiecie
\textsc{Matlab}, implementaującej algorytm największego spadku
do poszukiwania minimów, na przykładzie funkcji benchmarkowej Rosenbrocka.
Dodatkowym założeniem dla projektowanej aplikacji była maksymalna elastyczność w
postaci prostej podmiany minimalizowanej funkcji czy wyboru liczby wymiarów.

\section*{Wstęp}

Metoda największego spadku jest prostą iteracyjną metodą gradientową
poszukiwania minimów funkcjonałów określonych nad ciałem liczb rzeczywistych:

\begin{equation*}
    f: \mathbb{R}^{n} \rightarrow \mathbb{R} 
\end{equation*}

Dodatkowymi założeniami gwarantującymi poprawną zbieżność metody (gwarancja
zbieżności do minimum globalnego) są:

\begin{itemize}
  \item $f \in \mathcal{C}^{1}$ (funkcja ciągła i różniczkowalna)
  \item $f$ jest ściśle wypukła w badanym otoczeniu minimum globalnego
\end{itemize}

Jako że funkcja Rosenbrocka nie jest funkcją wypukłą na całej przestrzeni
$\mathbb{R}^{n}$ metoda nie gwarantuje zbieżności do minimum globalnego (dla
pewnych punktów startowych algorytm może zbiegać do minimum lokalnego, lub
punktu przegięcia funkcjonału).

\section*{Algorytm metody największego spadku}

Badana metoda jest metodą gradientową, iteracyjną, co oznacza że kolejne
przybliżenie minimum funkcji jest wyznaczane krokowo na podstawie poprzedniego
punktu oraz wartości gradientu w danym punkcie, co można przedstawić za pomocą
rekurencyjnego równania:

\begin{equation}
    x_{k+1} = \min_{\alpha \in [0, \alpha_{\textrm{m}}]} \left[x_{k} - \alpha \nabla f(x_{k})\right]
\end{equation}

Zatem problem minimalizacji funkcjonału w przestrzeni wielowymiarowej zostaje
sprowadzony do wielokrotnej minimalizacji funkcji jednej zmiennej $\alpha$.


\end{document}